% En formato APA 
\chapter*{Referencias Bibliográficas\markboth{Referencias Bibliográficas}{Referencias Bibliográficas}}

\begin{itemize}
    \item \textbf{AGUIRRE Jorge} (2013). \textit{Nuevos alcances de la participación ciudadana a través de las redes sociales.} Disponible en \small{http://www.scielo.org.mx}. Fecha de consulta: 30 de octubre 2014.

    \item \textbf{AMARAL José, with significant contributions from Michael Buro, Renee Elio, Jim Hoover, Ioanis Nikolaidis, Mohammad Salavatipour, Lorna Stewart, and Ken Wong} (2006), \textit{About Computing Science Research Methodology}. Disponible en

        \small{http://www.philadelphia.edu.jo/academics/startir/uploads/2013\-2/750791/Mourad\_Talk.ppt}.

    \item \textbf{BARCHINI Graciela E.} (2005).\textit{Métodos ``I+D'' de la Informática.} Revista de Informática Educativa y Medios Audiovisuales Vol 2(5), págs. 16-24. 2005. ISSN 1667-8338

  \item \textbf{BOEHM B. W.} (1988). \textit{A spiral model of software development and enhancement}. Computer, vol.21, no.5, pp.61,72, May 1988

  \item \textbf{BUNGE Mario} (2004).\textit{ La Ciencia. Su método y su filosofía. }Ediciones Siglo XX. Buenos Aires.

  \item \textbf{KUHN Thomas} (1962). \textit{The structure of scientific revolutions. }Fondo de la Cultura Económica.


\end{itemize}


