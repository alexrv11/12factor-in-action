\chapter{Marco teórico}
\noindent En el presente capítulo se conceptualizan términos relacionados al Desarrollo de software y que irán utilizando en los capítulos posteriores.

\section{Ingeniería de software}
\noindent La ingeniería de software está enfocado a desarrollo de software, por lo cual es necesario tomarlo en cuenta. Según [Roger S. Pressman, 2010] “La ingeniería de software abarca un proceso, una colección de métodos (práctica) y una gran variedad de herramientas que permiten a los profesionales construir software de alta calidad.”


\section{Proceso de desarrollo de Software}
\noindent El proceso de desarrollo define un marco de trabajo, donde se tiene como entrada las necesidades o requerimientos de los clientes y dentro del proceso se tiene un conjunto de actividades que nos permiten llegar a construir el producto.

\noindent Entre las principales actividades que se tiene en el proceso de desarrollo segun [Sommerville, 2005] son:
\begin{itemize}
  \item Especificación del Software: Donde los clientes y personas involucradas definen los requerimientos y sus restricciones del  Software.
  \item Desarrollo del Software: Donde el software se diseña y construye.
  \item Validación del Software: Donde se valida de que el software desarrollado cumple con las exigencias y expectativas del cliente.
  \item Evolución del Software: Donde el software tiende a cambiar y adaptarse a los nuevos requerimientos del cliente.
\end{itemize}

\begin{figure}[ht]
  \centering
  \includegraphics[width=12cm, height=4cm]{chapter2-process-software-development.png}
  \caption{Descripcion grafica del proceso de desarrollo [Elaboración propia]}  
\end{figure}

\noindent El proceso de desarrollo del software tiene que ir mejorando en conjunto al desarrollo del producto, tiene que ir solucionando las debilidades o necesidades que se pueden tener a medida que se va desarrollando el producto.
 
\section{Modelo del Proceso de Desarrollo de software}
\noindent El Modelo del proceso de desarrollo de software es una representación del proceso que se quiere aplicar para el desarrollo del software. En el modelo se define y se puede ver claramente el flujo de trabajo e interacción de las personas involucradas en cada actividad o etapa dentro del desarrollo de software.
\noindent Los Modelos pueden incluir actividades que son parte de los procesos y productos de software y el papel de las personas involucradas en la ingeniería del software. Algunos de los modelos que se pueden tener segun [Sommerville, 2005] son:
\begin{itemize}
\item Un modelo del flujo de trabajo: En este modelo se tiene el flujo de actividades humanas en el proceso junto a sus entradas, salidas y dependencias.
\item Un modelo del flujo de datos o actividades: En este modelo se tiene el conjunto de actividades y el tratamiento de datos que realiza cada actividad.
\item Un modelo de roles:En este modelo se tiene el conjunto de roles y responsabilidades de las personas involucradas en el proceso de desarrollo.
\end{itemize}

\noindent Los modelos de proceso de desarrollo pueden variar de acuerdo a los siguientes factores:
\begin{itemize}
\item Enfoque de desarrollo: Modelo en cascada, espiral, iterativo e incremental.
El tipo de software que se quiere desarrollar: aplicación de escritorio, plugin, aplicación web de propósito general.
\item mantenimiento de versiones: Si el software en desarrollo va a tener versiones o solo sera una unica version al mismo tiempo.
\item manifiesto de paradigmas de programación: Se puede tener modelos que estén reforzados con manifiesto o metodologías para el desarrollo del software.
\end{itemize}
\noindent Se debe tener mucho cuidado en adaptar un modelo, no se puede forzar el uso del modelo dentro el desarrollo del software.   

\section{Confiabilidad, Escalabilidad y Mantenimiento}
\noindent Los productos de software segun [Martin Kleppmann, 2014] tienen un cierto número de atributos asociados que reflejan la calidad de ese software. Estos atributos no están directamente asociados con lo que el software hace, Más bien, reflejan su comportamiento durante la ejecución, la estructura y organización del código fuente y la documentación asociada.

\subsection{Confiabilidad}
\noindent La confiabilidad se da en las interacciones que realiza el usuario final en la aplicación y que la aplicación pueda responder a las expectativas del usuario. La aplicación tiene que ser tolerante a fallos o a posibles errores que pueda cometer el usuario, debe poder responder de forma adecuada a situaciones inesperadas. El tiempo de respuesta debe ser lo más razonablemente posible. La seguridad en el acceso a información de los usuarios debe ser controlada, de tal forma que el usuario sepa que su información está segura y que las personas autorizadas puedan entrar.


\subsection{Escalabilidad}
\noindent La escalabilidad es la habilidad del software desarrollado, de poder adaptarse a incrementos de carga. Para decir que un software es escalable, se debe analizar su comportamiento con cierto número de usuarios que utilizan el software de forma simultánea e ir incrementando el número de usuarios para conocer la variación que se da en cuanto a los tiempos de respuesta y como se comporta el software en las diferentes situaciones.
\noindent Para poder tener una aplicación escalable, se debe definir el patrón de arquitectura a utilizar. La interacción de las capas definidas en el software, deben ser lo más vertical posible. Se debe tener un balance de carga que optimiza el uso de recursos de hardware. 

\subsection{Mantenimiento}
\noindent El costo del software no se refleja al inicio de desarrollo del software, sino en la etapa de mantenimiento, arreglando errores, manteniendo el software en operación, investigando fallas, adaptando el software a nuevas plataformas tecnológicas, pagando deudas técnicas y agregando nuevas características.
Para reducir el costo y la dificultad de mantenimiento de un software, se debe tomar particular atención a tres principios de diseño para un software.
\begin{itemize}
\item Operabilidad: la puesta en ejecución, monitorio, manejo de logs deben de lo más fácil posible para la administración operativa del equipo.
\item Simplicidad: Los nuevos desarrolladores deberían de poder integrarse al proyecto y comprenderlo sin complicaciones.
\item Plasticidad:Los ingenieros en el futuro puedan hacer cambios en el software, de tal forma que solo se modifique la parte que se quiere (cambiar de tecnología o módulo).
\end{itemize}

\section{Integración Continua}
\noindent Una definicion de Integracion Continua por [Martin Fowler, 2006] \"Integración Continua es una buena práctica en el desarrollo de software, donde los miembros del equipo integran su trabajo con frecuencia, lo cual conduce a múltiples integraciones por dia\". Cada integración se verifica por una construcción automatizada que valida los cambios realizados a través de criterios de validación como ser:
\begin{itemize}
\item Ejecución de pruebas automatizados
\item Código estándar
\item porcentaje de Cobertura de las pruebas unitarias en el código fuente.
\end{itemize}

\noindent El valor que se tiene con Integración Continua en el desarrollo de software son:
\begin{itemize}
\item Reducir de riesgos: Integrando varias veces al día, podemos reducir riesgos en la facilidad de encontrar problemas o errores generados recientemente. los errores son encontrados y arreglados lo mas antes posible.
\item Reducir procesos manuales repetitivos: Reduciendo procesos repetitivos, podemos ahorrarnos tiempo, costos y esfuerzo.
\item Generar software entregable en cualquier momento y cualquier lugar: Tener software entregable en cualquier momento es lo más tangible para los clientes o usuarios. que pueden percibir las nuevas características del producto de software.
\item Una mejor Visibilidad del proyecto: Tener la posibilidad de observar las tendencias de proyecto y tomar decisiones efectivas. Integración Continua nos proporciona información del estado de las métricas de calidad del producto de software.
\item Establecer una mayor confianza en el proyecto: Por cada ejecución de la Integración Continua, podemos conocer el impacto de los cambios realizados dentro en proyecto.
\end{itemize}

\section{Software como un Servicio}
\noindent Si el software a desarrollo es de proposito general y esta enfocada a diferentes usuarios de un mismo rubro empresarial o personal. Se puede reducir los precios de accesibilidad al producto software, ya que puede ser compartido por todos los usuarios que van a usar el software implementandolo como un servicio en internet.

\noindent El software como servicio (SaaS) es un modelo de distribución de software en el que las aplicaciones están alojados por un proveedor de servicio y puestos a disposición de los clientes en una red, normalmente Internet.

\noindent El término \"Software as a Service\" según [Saas manifiesto, 2010], se refiere más específicamente a software de negocios que se ejecuta en la nube. Las aplicaciones SaaS típicamente permiten a los clientes licenciar el software y el soporte que desee utilizar sin necesidad de instalar o mantener ningún software o hardware. En otras palabras, el vendedor proporciona un servicio de que puede ser suscrito y acceder a través de Internet en lugar de un producto físico que los clientes tienen que instalar y gestionar por su cuenta.

\subsection{Beneficios de implementar un SaaS}
\noindent Los beneficios que se pueden tener al aplicar un SaaS como solución tanto para el proveedor y los clientes son los siguientes.
\noindent Beneficios para el proveedor del Servicio:
\begin{itemize}
\item Al tener centralizado el software en la nube no tiene la necesidad de tener un plan de distribución de software.
\item La versión es solo uno para todos los clientes, lo cual no es necesario tener soporte técnico para versiones anteriores de software.
\item Bajos costos de desplegar el software a una nueva versión.
\end{itemize}

\noindent Beneficios para los clientes del Servicio:
\begin{itemize}
\item Bajos costos destinados a sus Centros de Tecnología e Información TI.
\item Tener la última versión del software lo cual significa tener nuevas características y problemas anteriores solucionados respecto a la versión anterior.
\item Acceder al Servicio en cualquier momento y desde cualquier lugar con acceso a Internet.
\end{itemize}

\section{Ambientes de ejecución}
\noindent Los entornos de ejecución para producción, son muy importantes para la ejecución del producto software. Con buenas características de hardware, los servidores pueden mejorar el rendimiento y tiempos de respuesta de las interacciones de los usuarios con el software. Se puede tener diferentes opciones para tener ambientes de ejecución.

\subsection{Servidores Propios}
\noindent La ventaja de tener servidores propios, la administración es de forma directa. Tener servidores propios hacen que se eleven los costos del producto de las siguientes maneras:
\begin{itemize}
\item Costos de instalación del servidor, donde se ejecuta el software para producción. Dependiendo a las características de hardware y el rendimiento que se quiere tener este costo puede elevarse. En el caso de que es un nuevo software y se desconoce el crecimiento de usuarios con el tiempo, se puede dar que se instala un servidor con buenas características de hardware pero la inversión no rinde con el número de usuarios que podrían ser menos de los planificado. Se puede tener el otro caso, se invierte en lo mínimo en características de hardware para que inicie el software. pero con el tiempo el número de usuarios incrementa de tal forma que es necesario incrementar las características de hardware lo cual hace que se invierta nuevamente en hardware.
\item Costos de ambientes donde se instalará los servidores, lo cual ocupa un espacio.
\item Costos de mantenimiento y monitoreo de los servidores. 
\end{itemize}

\subsection{Servicios en la Nube} 
\noindent La tendencia de usar servicios en la nube, que proporcionen entornos virtuales de ejecución con las características necesarias es una buena opción con las siguientes ventajas:
\begin{itemize}
\item los servicios en la nube que proporcionan entornos de ejecución dan la opción de planes de servicios. Si la aplicación está iniciando y se desconoce el crecimiento en el mercado, se puede elegir un plan mínimo para iniciar. Con el tiempo se puede analizar y ver si es necesario incrementar los rendimientos en cuanto al número de usuarios, lo único que se haria seria cambiar de plan.
\item La administración de los entornos de ejecución se reducen a monitorear la aplicación. El rendimiento que se tiene respecto a las interacciones de los usuarios.
\end{itemize}

\noindent En este capítulo se enfoca en enfatizar los factores que se deben considerar en el desarrollo de software hasta la puesta en producción del producto desarrollado. Al inicio del desarrollo se debe conocer las expectativas que tienen los clientes respecto al producto que se quiere desarrollar, de tal forma que el equipo de desarrollo direccione lo necesario para cumplir con las expectativas. La comunicación cliente - equipo de desarrollo debe ser lo más fluida posible.
\noindent El proceso de desarrollo debe iniciar con lo básico necesario, pero no debe ser impuesta sino debe ir adaptándose según las necesidades que se van teniendo en el transcurso del desarrollo. 